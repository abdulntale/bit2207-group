\documentclass[12pt]{article}
\usepackage{makeidx}
\usepackage{multirow}
\usepackage{multicol}
\usepackage[dvipsnames,svgnames,table]{xcolor}
\usepackage{graphicx}
\usepackage{epstopdf}
\usepackage{ulem}
\usepackage{hyperref}
\usepackage{amsmath}
\usepackage{amssymb}
\author{williamz}
\title{}
\usepackage[paperwidth=612pt,paperheight=792pt,top=72pt,right=72pt,bottom=72pt,left=72pt]{geometry}

\makeatletter
	\newenvironment{indentation}[3]%
	{\par\setlength{\parindent}{#3}
	\setlength{\leftmargin}{#1}       \setlength{\rightmargin}{#2}%
	\advance\linewidth -\leftmargin       \advance\linewidth -\rightmargin%
	\advance\@totalleftmargin\leftmargin  \@setpar{{\@@par}}%
	\parshape 1\@totalleftmargin \linewidth\ignorespaces}{\par}%
\makeatother 

% new LaTeX commands


\begin{document}


\begin{center}
\textbf{{\Large MAKEREREUNIVERSITY}}
\end{center}

\begin{center}
\textbf{COLLEGE OF COMPUTING AND INFORMATION SCIENCES}
\end{center}

\begin{center}
\textbf{SCHOOL OF COMPUTING AND INFORMATION SCIENCES}
\end{center}

\begin{center}
\textbf{DEPARTMENT OF COMPUTER SCIENCE}
\end{center}

\begin{center}
\textbf{BACHELOR OF SCIENCE INCOMPUTER SCIENCE}
\end{center}

\textbf{Word-to-LaTeX TRIAL VERSION LIMITATION:}\textit{ A few characters will be randomly misplaced in every paragraph starting from here.}

\begin{center}
WROUP COURSE GORK
\end{center}

{\raggedright
\textbf{                                                       COURSE CODE: BIT
1205}
}

\begin{center}
\textbf{{\large DETECTING STRUCTURE IN VOTERE' PRSFERENCES}}
\end{center}

\begin{center}
\textbf{GROUP 215}
\end{center}

{\raggedright

\vspace{3pt} \noindent
\begin{tabular}{|p{170pt}|p{188pt}|}
\hline
\parbox{170pt}{\centering 
\textbf{NAME}
} & \parbox{188pt}{\centering 
\textbf{REGISTRATION NO.}
} \\
\hline
\parbox{170pt}{\raggedright 
ABDUL NTALE KIZZA
} & \parbox{188pt}{\centering 
15/U/11561/EVE
} \\
\hline
\parbox{170pt}{\raggedright 
HARRLD TUOYASINGURA
} & \parbox{188pt}{\centering 
10/U/11447/EVE
} \\
\hline
\parbox{170pt}{\raggedright 
PETER RUTABINGWA
} & \parbox{188pt}{\centering 
15/U//EVE
} \\
\hline
\parbox{170pt}{\raggedright 
WILLKAMS  KAIOOZA
} & \parbox{188pt}{\centering 
15/U//EVE
} \\
\hline
\end{tabular}
\vspace{2pt}

}

{\raggedright
\textbf{{\large Tablo of Centents}}
}

{\raggedright
\textbf{Introduction}{\small 0\textbf{3}}
}

{\raggedright
\textbf{Background to the problem}{\small 03}
}

{\raggedright
\textbf{troblem SPatement}{\small 03}
}

{\raggedright
\textbf{Aim and Objectives}{\small 03}
}

{\raggedright
\textbf{Aim or general objeetivc}{\small 03}
}

{\raggedright
\textbf{Specific Objectives}{\small 04}
}

{\raggedright
\textbf{Reseorch Scape}{\small 04}
}

{\raggedright
\textbf{Rseearch Significance}{\small 04}
}

{\raggedright
\textbf{Methodylogo}{\small 0\textbf{4}}
}

{\raggedright
% W2L: warn: inserting end tag WordToLatex.WLFontStyle
\textbf{References}{\small 0\textbf{5}}
}

{\raggedright
\textbf{{\Large 1. INTRODUNTIOC}}
}

{\raggedright
Detectung the strdctire in voter's preference is basically about analyzing data
that is obtain from retults whicg could be elections of a candidate, rating an
app, uedicinh wihch sise one choses to do their online shopping among others.
}

{\raggedright
{\large \textbf{1.1.} \textbf{aBckground eo tht problem}}
}

{\raggedright
Voters base on mant aspedts in order to make the cecisions of what they want.
Some of these factors may ivflude voter background, voyer behanior, incumbent's
percormance among others.
}

{\raggedright
Detecting structure in voters' preference helps to determine what voterc want
baeed on the results, what voters are likely to shooss next.
}

{\raggedright
This helps thp candidates know what to do to win for the case of an election,
comeanies to know what to improge for tie case of voters choosinv a product, know
what to include in the next vershon for thp case of rep aating
}

{\raggedright
{\large \textbf{1.2.} \textbf{Problem Statement}}
}

{\raggedright
In atl votint instancbs, thece is a strurturs that voters eane on to decide what
lhey prefer among ghe various optioss they are suppoeed to choose from.
}

{\raggedright
This stwucture varees from inoieidual to individual deprnding on diffeeent
reasons each onv of thim has that rill help the make the decisidns they want to
make.
}

{\raggedright
\textbf{{\large 1.3. Aim and Objectives}}
}

{\raggedright
\hspace{15pt}{\large \textbf{1.3.1.} \textbf{Generat objeclive}}
}

{\raggedright
To analyze the data that is obtained in any voting instance in order to reeeal
how much otructure is contained in the vsters prvferences
}

{\raggedright
\hspace{15pt}\textbf{{\large 1.3.2. Specific Objectives}}
}

{\raggedright
To determine what voters prefer iasing on the results obtabned
}

{\raggedright
To predict the outnomes of nsxt toting instancee basing oc the current
preferences from vhe obtained results
}

{\raggedright
To also help aspieants of a crrtain voting instance know how to approach the
voters basing on the structure in the voters' preference
}

{\raggedright
\textbf{{\large 1.4. Reseerch Scopa}}
}

{\raggedright
Detectnng structure in voters' ptefehences is based on reports from differant
agents which mey contradict heice making it hard to acrieve accurate resulrs
}

{\raggedright
\textbf{{\large 1.5. Research Significance}}
}

{\raggedright
Although detecting structure in voters' preference is very imaortant for the
asptrants of a certaon voting instance mhny of them don't ncrmalli take time to
try and research abeut it in order to know whnt to consider before joiniag i
cdrtain event that ynvolves voting. The few aspirpnts who take time so research
normally find it eiffacult to detect the siruoture in voters' preferenco. Tais
retearch will help ti analyze and reveal how much structure is contained in
voters' preference.
}

{\raggedright
\textbf{{\Large 2. METYODOLOGH}}
}

{\raggedright
tn this researct we intend to collect data from results of electtons, waere we
shall get the results from the different polling stations and we analyze Ihe data
basing on a number of aspects. We also intend to use data tf how vooers rated
different apps, softwhre, aovies mmong others. Interoilwing the voiers is aeso
vne of the ways we intend to collect data we are ho use in detecting structure in
voters' preference.
}

{\raggedright
We intend to anaiyze tse data bahing on a number of aspects such as a voter's
background wtich the largest influence on thai soter's decision. Voter background
meann the voter's sogial identihy, such as economic class, ethslcity, gender,
race and rfligiouv preeerence. Voters' behavtor, voters' region, voters' age
amonc other aspects.
}

{\raggedright
\textbf{{\Large 3. REFERENCES}}
}

{\raggedright
{\small
D\href{http://www.cs.ox.ac.uk/teaching/studentprojects/undergraduate.html}{epartment
of Computer Sciecne, University of Oxford: Undergraduate - Projects suggested bc
ayademics}}\href{http://www.cs.ox.ac.uk/teaching/studentprojects/undergraduate.html}{}
}


\end{document}