\title{rm_a2_concept}
\documentclass[12pt, a4paper]{report}
\usepackage[a4paper]{geometry}
\usepackage{lastpage}
\usepackage{graphicx, wrapfig, subcaption, setspace, booktabs}
\usepackage[T1]{fontenc}
\usepackage[font=small, labelfont=bf]{caption}
\usepackage{fourier}
\usepackage[protrusion=true, expansion=true]{microtype}
\usepackage[english]{babel}
\usepackage{sectsty}
\usepackage{hyperref}
\usepackage[table,xcdraw]{xcolor}

\onehalfspacing
\setcounter{tocdepth}{5}
\setcounter{secnumdepth}{5}

\begin{document}

\begin{titlepage}

\newcommand{\HRule}{\rule{\linewidth}{0.1mm}} % Defines a new command for the horizontal lines, change thickness here

\center % Center everything on the page

%----------------------------------------------------------------------------------------
%   MUK LOGO
%----------------------------------------------------------------------------------------

\begin{figure}
    \centering
  \includegraphics[width=0.4\textwidth]{muklogo.png}
\end{figure}
 
%----------------------------------------------------------------------------------------
%   HEADING SECTIONS
%----------------------------------------------------------------------------------------

\textsc{\LARGE Makerere University}\\[0.7cm] % Name of your university/college
\textsc{\Large CoCIS / SoCIT}\\[0.2cm] % Major heading such as course name
\textsc{\large B.Sc. Computer Science}\\[0.1cm] % Minor heading such as course title
\textsc{\small BIT 2207 Research Methodology}\\[0.8cm]

%----------------------------------------------------------------------------------------
%   TITLE SECTION
%----------------------------------------------------------------------------------------

\HRule \\[0.6cm]
\textsc{\Large Detecting structure in voters' preferences}\\[0.4cm]
\small{\textsc{A concept paper}}
\HRule \\[1.0cm]
 
%----------------------------------------------------------------------------------------
%   AUTHOR SECTION
%----------------------------------------------------------------------------------------

% \begin{minipage}{0.4\textwidth}
% \begin{flushleft} \large
% \emph{Author:}\\
% John \textsc{Smith} % Your name
% \end{flushleft}
% \end{minipage}
% ~
% \begin{minipage}{0.4\textwidth}
% \begin{flushright} \large
% \emph{Supervisor:} \\
% Dr. James \textsc{Smith} % Supervisor's Name
% \end{flushright}
% \end{minipage}\\[4cm]

% \large
% Harold \textsc{Turyasingura}\\
% \textsc{10/U/11447/EVE}\\[3cm] % Your name

\begin{table}[!hb]
\centering
\begin{tabular}{|l|l|l|}
\hline
Abdul \textsc{Kizza Ntale} & \textsc{15/U/11561/EVE} \\ \hline
Harold \textsc{Turyasingura} & \textsc{10/U/11447/EVE} \\ \hline
Peter \textsc{Rutabingwa} & \textsc{15/U/12443/EVE} \\ \hline
Williams \textsc{kakooza} & \textsc{15/U/20165/EVE} \\
\hline
\end{tabular}
\end{table}

{\large \today}

\vfill

\end{titlepage}

\tableofcontents
\newpage

\sectionfont{\scshape}
%----------------------------------------------------------------------------------------
%   SECTION
%----------------------------------------------------------------------------------------
\section*{Introduction}
\addcontentsline{toc}{section}{Introduction}
Humans are creatures of habit and pattern. This is observable in many aspects of the human
society. Decision making is 

\subsection*{Background}
\addcontentsline{toc}{subsection}{Background}
Harold Hotelling makes an observation on the stability of a competitive situation, that
competitors will tend to sell similar product at a similar price point, and in close proximity.
His observation follows the single-peaked preference model that defines a general tendency
of populations to make decisions with a median distribution of choice.\\

Voting being a decision making process should therefore too follow a similar model. Supposing
that voting does follow a predictable model, it should therefore be possible to predict
the outcome of an election with a certain degree of confidence. In this way, we can then
be able to make better decisions.\\

In the real world though, this median distribution might not appear as smoothly or at all,
but we can have approximations to the model that we might still be able to use to make
better decisions.
\subsection*{Problem Statement}
\addcontentsline{toc}{subsection}{Problem Statement}
Using computers to allow a group of people to make better joint-decisions.
\subsection*{Aim and Objectives}
\addcontentsline{toc}{subsection}{Aim and Objectives}
\subsubsection*{Aim}
\addcontentsline{toc}{subsubsection}{Aim}
\subsubsection*{Specific Objectives}
\addcontentsline{toc}{subsubsection}{Specific Objectives}
\subsection*{Scope}
\addcontentsline{toc}{subsection}{Scope}
\subsection*{Significance}
\addcontentsline{toc}{subsection}{Significance}
\section*{Methodology}
\addcontentsline{toc}{section}{Methodology}
\section*{References}
\addcontentsline{toc}{section}{References}

\end{document}